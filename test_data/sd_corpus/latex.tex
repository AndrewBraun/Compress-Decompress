\documentclass[10pt,a4paper]{article}
\author{Jodie Foster}
%\date{\today}
\title{Ship Maintenance 436, Assignment 2}
\usepackage[utf8]{inputenc}
\usepackage{amsmath}
\usepackage{mathtools}
\usepackage{amsfonts}
\usepackage{amssymb}
\usepackage[margin=1in]{geometry}
\usepackage{graphicx}
\usepackage{hyperref}
\usepackage{yfonts}
\usepackage{units}
\usepackage{float}
\usepackage{epstopdf}
\usepackage{tikz-cd}
\usepackage{subfig}
\DeclareGraphicsRule{.tif}{png}{.png}{`convert #1 `dirname #1`/`basename #1 .tif`.png}
\newcommand{\degr}[1] {\text{deg}_p(#1)}
\newcommand{\ord}[2] {\text{ord}_{{#2}}(#1)}
\newcommand{\mr}{\mathbb{R}}
\newenvironment{definition}
{ \rule{1ex}{1ex}\hspace{\stretch{1}} }
{ \hspace{\stretch{1}}\rule{1ex}{1ex} }

\def\Xint#1{\mathchoice
{\XXint\displaystyle\textstyle{#1}}%
{\XXint\textstyle\scriptstyle{#1}}%
{\XXint\scriptstyle\scriptscriptstyle{#1}}%
{\XXint\scriptscriptstyle\scriptscriptstyle{#1}}%
\!\int}
\def\XXint#1#2#3{{\setbox0=\hbox{$#1{#2#3}{\int}$ }
\vcenter{\hbox{$#2#3$ }}\kern-.6\wd0}}
\def\ddashint{\Xint=}
\def\dashint{\Xint-}

\linespread{1.5}
\begin{document}
\maketitle
\section*{Question 1}
The following diagram commutes:
\begin{center}
\begin{tikzcd}
X_1 \arrow[r, "f_1"] \arrow[d, "\psi"]
& X_1 \arrow[d, "\psi"] \\
X_2 \arrow[r, "f_2"]
& X_2
\end{tikzcd}\end{center}
Let $E\subseteq X_2$. Since all three of our functions are measurable, all the preimages we will encounter will be measurable. Now, we have that \begin{align*}\psi_*\mu(f_2^{-1}(E))&=\mu(\psi^{-1}\circ f_2^{-1}(E))\\&=\mu((\psi\circ f_2)^{-1}(E))\\&=\mu((f_1\circ \psi)^{-1}(E))\\&=\mu((f_1^{-1}(\psi^{-1}(E)))\\&=\mu(\psi^{-1}(E))\\&=\psi_*\mu (E)\end{align*}
Since this holds for all measurable $E\subseteq X_2$, $\psi_*\mu$ is $f_2$-invariant.
\section*{Question 2}
We prove this, and that $f^k$ is a measurable function, by induction on $k$. The base case, $k=1$, is simply the hypothesis. Suppose it holds for some $k$; then if $E$ is measurable, so is $(f^k){-1}(E)$, and therefore $(f^{k+1})^{-1}(E)=f^{-1}((f^k)^{-1}(E))$ is also measurable. Therefore $f^{k+1}$ is a measurable function. To finish the proof, simply note that $\mu((f^{k+1})^{-1}(E))=\mu(f^{-1}((f^k)^{-1}(E)))=\mu((f^k)^{-1}(E))$ by $f$-invariance of $\mu$, but this is $\mu(E)$ by the induction hypothesis.\\\\
The converse, however, is not true. Take for instance the map $f:[0,1)\to[0,1)$ that takes $x$ to $x+0.5\pmod{1}$. It clearly squares to the identity ($f^2(x)=x+1\pmod{1}$ is simply $x$), and of course the identity is $\mu$-invariant for any $\mu$. But let $\mu$ be the measure giving point mass to 0. Then $\mu(\{0\})=1$ but $\mu(f^{-1}(\{0\}))=\mu(\{0.5\})=0$.
\section*{Question 3}
\subsection*{Part A}
Let us define $f$ to be a function from $(0,1)$ to itself, and to take $1/n$ to, say, 0.5 for each integer $n$. Since the set of $1/n$'s is negligible, $f$ is a piecewise diffeomorphism from $(1/(n+1),1/n)$ to (0,1) (and in fact is a local diffeomorphism from countably many intervals in the domain), we have that the transfer operator $P_f$ (or simply $P$) takes $\rho$ to $\displaystyle \sum_{i=1}^\infty \frac{\rho\circ f_i^{-1}}{|f'\circ f_i^{-1}|}\chi_{f_i(E_i)}$, where $f_i$ is $f$ restricted to the domain $E_i=(1/(i+1),1/i)$.\\\\
To actually compute this map, notice that $f'$ on $E_i$ is the constant function $i(i+1)$, $f_i^{-1}(y)=\frac{y}{(i+1)i}+\frac{1}{i+1}$, and therefore
$$ P\rho(y)=\sum_{i=1}^\infty \frac{1}{(i+1)i}\rho\left(\frac{y}{(i+1)i}+\frac{1}{i+1}\right)$$
We must first prove that Lebesque is invariant under $f$. As a good first step, we will show invariance when $E$ is an interval $(a,b)\subseteq(0,1)$ (either end of the interval may be open or closed; it does not matter since both a single point and the $f$-preimage of a point has zero measure). The $f$-preimage of $(a,b)$ is the union of each $f_i$-preimage (modulo, perhaps, a 0-measure set, the preimage of $\{0.5\}$):
$$f^{-1}(a,b)=\bigcup_{i\in\mathbb{N}} f_i^{-1}(a,b)=\bigcup_{i\in\mathbb{N}} (f_i^{-1}(a),f_i^{-1}(b))=\bigcup_{i\in\mathbb{N}} \left(\frac{a}{i(i+1)},\frac{b}{i(i+1)}\right)$$
where the second-last equality holds from the fact that $f_i^{-1}(a)<f^{-1}(b)$ and that $f^{-1}_i$, being continuous, sends intervals to intervals. Therefore, the Lebesque measure of $f^{-1}(a,b)$ is
\begin{align*}\sum_{i=1}^\infty \text{Leb}\left(\frac{a}{i(i+1)},\frac{b}{i(i+1)}\right)&=\sum_{i=1}^\infty\frac{b-a}{i(i+1)}\\&=(b-a)\sum^\infty_{i=1}\left(\frac{1}{i}-\frac{1}{i+1}\right)\\&=b-a=\text{Leb}(a,b)\end{align*}
where we used telescoping series to evaluate the summation. Also, since $f$-invariance of Lebesque holds for intervals, it holds for finite unions of intervals (no new sets are gained by taking finite intersections). We therefore have $f$-invariance of Lebesque on the algebra generated by intervals. To show that this extends to the Borel $\sigma$-algebra on $(0,1)$, we need to invoke Carath\'eodory. (Indeed, if we invoke his name three times, he will appear, like Beetlejuice.)\\\\
Consider that checking $f$-invariance of Lebesque is exactly the same as checking that $f_*\text{Leb}=\text{Leb}$; we have shown that these are the same on the algebra generated by the intervals. Since Carath\'eodory \textit{uniquely} extends to a $\sigma$-algebra, we have that $f_*\text{Leb}=\text{Leb}$ on the Borel algebra. Therefore, $\text{Leb}(f^{-1}(E))=\text{Leb}(E)$ for any measurable $E\subseteq(0,1)$, and we are done.\\\\
Next, to show that Lebesque is the unique invariant measure that is absolutely continuous with respect to Lebesque, we require a preliminary result: I claim that if $\rho$ is Lipschitz on $(0,1)$, then $\|P\rho\|_{\text{Lip}}\leq C\|\rho\|_{\text{Lip}}$ for some $0<C<1$. For take such a $\rho$, and any $x,y$:
\begin{align*}\frac{|P\rho(x)-P\rho(y)|}{|x-y|}&=\left|\sum_{i=1}^\infty \frac{1}{(i+1)i}\left(\rho\left(\frac{x}{(i+1)i}+\frac{1}{i+1}\right)-\rho\left(\frac{y}{(i+1)i}+\frac{1}{i+1}\right)\right)\right|\cdot\frac{1}{|x-y|}\\
&\leq \sum_{i=1}^\infty \frac{1}{(i+1)i}\left|\left(\rho\left(\frac{x}{(i+1)i}+\frac{1}{i+1}\right)-\rho\left(\frac{y}{(i+1)i}+\frac{1}{i+1}\right)\right)\right|\cdot\frac{1}{|x-y|}\\
&\leq \frac{\sum_{i=1}^\infty \frac{1}{(i+1)i}\left|\left(\rho\left(\frac{x}{(i+1)i}+\frac{1}{i+1}\right)-\rho\left(\frac{y}{(i+1)i}+\frac{1}{i+1}\right)\right)\right|}{\frac{1}{|\left(\frac{x}{i(i+1)}+\frac{1}{i+1}\right)-\left(\frac{y}{i(i+1)}+\frac{1}{i+1}\right)|}\frac{1}{i(i+1)}}\\
&= \sum_{i=1}^\infty\frac{1}{i^2(i+1)^2}\|\rho\|_{\text{Lip}}= \|\rho\|_{\text{Lip}}\sum_{i=1}^\infty\frac{1}{i^2(i+1)^2}\end{align*}
Since this holds for all $x,y\in(0,1)$, and the summation term does not depend on the points chosen, it remains to show that $\displaystyle C:=\sum_{i=1}^\infty\frac{1}{i^2(i+1)^2}<1$. I have no idea what the summation converges to, but we have
\begin{align*}\sum_{i=1}^\infty\frac{1}{i^2(i+1)^2}&=\sum_{i=1}^\infty\left(\frac{1}{i^2}-\frac{1}{(i+1)^2}-\frac{2i}{i^2(i+1)^2}\right)\\&=\sum_{i=1}^\infty\left(\frac{1}{i^2}-\frac{1}{(i+1)^2}\right)-\sum_{i=1}^\infty\frac{2i}{i^2(i+1)^2}\\&=1-\sum_{i=1}^\infty\frac{2i}{i^2(i+1)^2},\end{align*}
by telescoping the first series, and so $c$ surely converges to something less than 1. And obviously since all summation terms in $c$ are positive, it converges to something positive.\\\\
We therefore get, by induction, that $\|P^n\rho\|_{\text{Lip}}\leq C^n\|\rho\|_{\text{Lip}}$, which goes to 0 with $n$, so that $P^n\rho$ converges to the 0 function. From here on out, I will simply regurgitate the argument given in the notes, rewording it so that I know I understand every detail. Feel free to skip it, if you like (the next page or so was written entirely for my own benefit).\\\\
Take an absolutely continuous (with respect to Lebesque) probability measure $\nu$ that is invariant with respect to $f$; by Radon-Nikodyn we have the existence of an integrable function $\rho$ such that for any measurable set $E\subseteq(0,1)$, $\nu(E)=\int_E \rho dx$. Since $\nu$ is a probability measure, $\rho-1$, besides being integrable, also integrates to 0: $\int_{(0,1)}\rho-1dx=0$. Also, since $\nu$ is invariant under $f$, $P\rho=\rho$ (by Exercise 4iv in the notes). Since Lebesque is also invariant under $f$, $P^n\rho-P^n1=\rho-1$ for any $n$.\\\\
If we can show that $\|\rho-1\|_{L^1}=0$, then clearly $\rho=1$ and $\nu$ is just Lebesque measure.\\\\
To do this, suppose $\|\rho-1\|_{L^1}=\epsilon>0$. First invoke density of Lipschitz functions in $L^1$ to find a Lipschitz function $h$ such that $\|\rho-h\|_{L^1}<\epsilon/3$. Also, find $n$ large enough that $C^n\|h-\int h\|_{\text{Lip}}<\epsilon/3$ (this can be done since $h-c$ is Lipschitz for any constant $c$, and since $0<C<1$).\\\\
Now we have, since $\int1=1$ (we are in a probability space), contractiveness of $P$, and the fact that $P^n\rho-P^n1=\rho-1$, that
\begin{align*}\|\rho-1\|_{L^1}&=\|P^n\rho-P^n1\|_{L^1}\\
&\leq\|P^n\rho-P^nh\|_{L^1}+\|P^nh-P^n1\|_{L^1}\\
&=\|P^n(\rho-h)\|_{L^1}+\|P^n(h-\smallint h)+P^n(\smallint 1-1)+P^n(\smallint h-\smallint 1)\|_{L^1}\\
&\leq\|\rho-h\|_{L^1}+\|P^n(h-\smallint h)+P^n(0)\|_{L^1}+\|P^n(\smallint h-\smallint 1)\|_{L^1}\\
&\leq\|\rho-h\|_{L^1}+\|h-\smallint h\|_{L^1}+\|\smallint h-\smallint 1\|_{L^1}
\end{align*}
We have also used the triangle inequality and linearity of $P$ freely throughout, and also the fact that everything is integrable.\\\\
Now this first term is $<\epsilon/3$. The third term can also be seen to be less than $\epsilon/3$: when we subtract 0 from inside the norm, we get $\|(\smallint h-\smallint 1)-\smallint(\rho-1)\|_{L^1}=\|(\smallint h-\smallint \rho)\|_{L^1}<\epsilon/3$. The middle term, however, takes a little work.\\\\
First notice that since $\smallint h$ is a constant, $h-\smallint h$ is Lipschitz. Also, since we are in a probability space, $\smallint\smallint h=\smallint h$, and so by linearity of the integral, $\psi:=h-\smallint h$ integrates to 0. Also, since the transfer operator preserves integration, $P^n\psi$ also integrates to 0, and  therefore, by the integral statement of the mean value theorem, $(P^n\psi)(x_0)=0$ for some $x_0\in(0,1)$. Now, for any $x\in(0,1)$ $|P^n\psi(x)|=|P^n\psi(x)-P^n\psi(x_0)|\leq\|P^n\psi\|_{\text{Lip}}|x-x_0|$, since the Lipschitz norm is the supremum of the slopes $|P^n\psi(x)-P^n\psi(x_0)|/|x-x_0|$. But this upper bound is itself bounded by $C^n\|h-\smallint h\|_{\text{Lip}}<\epsilon/3$.\\\\
We have, at long last, shown a contradiction, that $\epsilon=\|\rho-1\|_{L^1}<\epsilon/3+\epsilon/3+\epsilon/3$. The assumption that broke was that $\rho\neq 1$ and therefore, we are done.
\subsection*{Part B}
I claim that for any $i\in\mathbb{N}-\{1\}$, $f_i$ has a fixed point. For $f_i(x)=i(i+1)x-i$, and therefore $f_i(x)-x=i^2x-i$. When $x$ limits to $\frac{1}{i+1}$, $f(x)-x$ limits to $\frac{-1}{i+1}<0$; when $x$ limits to $\frac{1}{i}$, $f(x)-x$ limits to $1-\frac{1}{i}$ (which is strictly positive when $i>1$. Therefore there exist values on $x$ on the domain of $f_i$ for which $f_i(x)-x$ is positive, and others for which it is negative. By the intermediate value theorem there exists a fixed point in $[1/(i+1),1/i]$ for every $i>1$ (that is, $f$ has infinitely many fixed points). Point mass at a fixed point yields an invariant probability measure (as the preimage of a set will contain the fixed point exactly when the set itself contains the fixed point). We have found infinitely many $f-$ invariant probability measures on $(0,1)$. 
\section*{Question 4}
Consider the set $A\subseteq [0,1)$ consisting of all reals whose decimal expansion begins with $a_1a_2...a_m$; that is, $A=[0.a_1a_2,...a_m,0.a_1a_2,...(a_m+1))$. This has nonzero Lebesque measure (in fact, its Lebesque measure is $10^{-m}$). Consider the shift map $f$ that takes $0.a_1a_2...$ to $0.a_2a_3...$; the $f$-invariance of Lebesque was shown in class.\\\\
Since Lebesque on $[0,1)$ is a finite measure, by Poincar\'e's recurrence theorem, for almost every $x\in[0,1)$ there exist infinitely many values of $n$ for which $f^n(x)\in A$. Since $f^n(x)\in A$ if and only if its $n$-through-$n+m$th digits are $a_1a_2...a_m$, we see that almost every $x\in[0,1]$ contains infinitely many occurrences of the digits $a_1a_2...a_m$.\\\\
Should we require distinct (nonoverlapping) occurrences of these digits, it suffices to apply Poincar\'e's theorem to the map $g=f^m$, which is $\mu$-invariant by Question 2.
\section*{Question 5}
Suppose that $S$ has unbounded gaps, and construct the strictly increasing integer sequence $\{n_i\}$ inductively as follows: let $n_0=0$, and when $n_i$ is defined, find a positive integer $k$ such that $k$, as well as the $n_i$ integers following $k$, are not in $S$ (this is possible since $S$ has unbounded gaps); define $n_{i+1}=k+n_i$. Not only is $n_{i+1}-n_i\notin S$, but also for any $j<i+1$, $n_{i+1}-n_j\notin S$.\\\\
Now for any $n_i$, $n_j$, $i<j$, $\mu(T^{-n_i}A\cap T^{{-n_j}}A)=\mu(T^{-n_i}(A\cap T^{n_i-n_j}A))=\mu(A\cap T^{-(n_j-n_i)}A)=0$ since $n_j-n_i\notin S$ and $T$ is invariant. Therefore, we have a countably infinite collection of sets $\{T^{-n_i}A\}$ that are pairwise disjoint except on a negligible set. Since $T$ is invariant, each of these sets have measure $\mu(A)>0$.\\\\
Now a lemma is required, which you may have proved in class but I have forgotten: countable additivity holds when each pair in a sequence of sets $\{A_n\}$ has negligible intersection. For, define the sequence of sets $\{B_n\}$ by $\displaystyle B_n=A_n-\bigcup_{i< n} A_i$. These are pairwise disjoint, and $\displaystyle \bigcup_{i<\infty} A_i=\bigcup_{i<\infty}B_n$. We wish to show $\mu(B_n)=\mu(A_n)$ (this is trivial for $n=1$, so suppose $n>1$). One half of the equality, $\leq$, is obvious since $B_n\subseteq A_n$; for the other half, note that $\displaystyle A_n=B_n\cup\bigcup_{i<n}(A_n\cap A_i)$, so $\displaystyle \mu(A_n)\leq\mu(B_n)+\sum_{i<n}\mu(A_n\cap A_i)=\mu(B_n)$ since each term in the summation is 0. Therefore,
$$\mu\left(\bigcup_{i<\infty}A_i\right)=\mu\left(\bigcup_{i<\infty}B_i\right)=\sum_{i=1}^\infty\mu(B_i)=\sum_{i=1}^\infty\mu(A_i).$$
\\
Anyway, having shown the lemma, we see that
$$1\geq\mu\left(\bigcup_{i}T^{-n_i}A\right)=\sum_{i=1}^\infty \mu(T^{-n_i}A)=\sum_{i=1}^\infty \mu(A)=\infty$$
since $\mu(A)$ is some fixed positive number. This contradicts our claim that $S$ has unbounded gaps.
\end{document}